% what is LATEX?
% LATEX is a math typesetting program, which allows mathematicians to type mathematical equations easily

%Latex document has two parts. 
%First part is called Preamble and the second one is Document Environment


%documentclass tells about what kind of document you are making
% {article} is the most default choice, {beamer} is used for presentations
\documentclass{article}
\documentclass[letterpaper,11pt]{article}


%packages gives extra Latex commands that helps us to make the document
%{geometry} package helps us adjust margins
\usepackage[margin=1.25in]{geometry}
\usepackage[top=topmargin, bottom=bottommargin, left=leftmargin, right=rightmargin]{geometry}
%{amsmath,amssymb} are helpful math packages
\usepackage{amsmath,amssymb}
%{graphicx} is used to include images
\usepackage{graphicx}


%the document environment starts with
\begin{document}
%ends with
\end{document}



%=============Basic Text Manupulation============%

%inside [] you can pass optional arguments, such as font size. This would be the default font-size for document.
\documentclass[10pt]{article}

% But inside the document you can change the font size by using different commands
\tiny
\scriptsize
\footnotesize
\normalsize
\large
\Large
\LARGE 
\huge 
\Huge 


%=========== changing the Text sizes

%if you dont use the {} ,the font switches to your desired one until you change it back explicitly 
%using {} create a limited scope
{\LARGE followed by your text}
% other font size packages :extsizes, anyfontsize



% ===========Font styles

%turns your text into bold
\textbf{your text}

%turns into itallic
\textit{}

%underlines the text and is used for single words
\underline{}

%used for underlining long string of text
\usepackage{ulem}
\uline{your text}


% ============Text Emphasis
\emph{your text}

% ========== Font Families
% Roman
\textrm{your text}
% sans serif
\textsf{your text}
% typewriter text
\texttt{your text}


%============ Text Justification/Alignment
%By default it is 'Fully justified Text', the text occupies both sides without leaving any spaces

%Center Justified Text
\begin{center} .... \end{center}

%Left Justifies Text
\begin{flushleft} .... \end{flushleft}

%Right Justifies Text
\begin{flushright} .... \end{flushright}



% ================Line Breaks
%\\ is used to switch to the next line
\\

%\\[\baselineskip] is used to switch to next line +1
\\[\baselineskip]



% =================Indentation
% latex automatically indents new paragraphs, for new paragraph you need to seperate your text by blank line

% \noindent is used if you want to skip indentation
\noindent

%if you want to change the indent size, you can set it in preamble
\setlength{\parindent}{1cm}






% ===================Tabular environment
% {lcr} indicates three cols, with left justified, center justified, right justified
% \hline is used for horizontal line
% for vertical line use '|' in {|lc|r|}
\begin{tabular}{lcr}
    \hline
    text & text & text \\
    l    & c    & r 
    \hline
\end{tabular}




% ===================Basic Customizations========

% Paper size can be changed by passing it as an optional parameter to \documentclass[option]
\documentclass[option]{article}
% options available are
a4paper
a5paper
b5paper
letterpaper
legalpaper
executivepaper



% to make the entire document in landscape mode 
\usepackage[landscape]{geometry}


% to make part of the document in landscape mode
\usepackage{pdflscape}
\begin{landscape}
\end{landscape}


% to print page number in every page 
\usepackage{fancyhdr}
\pagestyle{fancy}
\fancyfoot[C]{page \thepage}


% forcing page Break
% stretched out the text to fill the page "soft" page break
\pagebreak 

% pushes the whitespace to the bottom "Hard' page break
\newpage



% =================User defined Commands===========

% in preamble put the command
\newcommand{\commandname}{definition}

% example
\newcommand{\R}{\mathbb{R}}
% now command \R gives Real number symbol, the same output as \mathbb{R}



% ==================Latex environments===============
% everything in between \begin and \end is inside a environment where it behaves in a particular fashion
\begin{document}
    \begin{center}
    \end{center}
\end{document}

% some of the environments are 
% document - sets up the basic document structure
% center - centers the text on the page
% flushright - right-justifies the text on the page
% flushleft - left justifies the text on the page
% tabular - a text table environment
% itemize and enumerate - both of them creates indented lists. Itemize command creates bullet points, where as Enumerate creates numbered list
    \begin{itemize}
        \item first item 
        \item second item 
        \begin{itemize}
            \item First sub-item 
            \item second sub-item 
        \end{itemize}
    \end{itemize}
    

    \begin{enumerate}
        \item first item 
        \item second item 
        \begin{enumerate}
            \item First sub-item 
            \item second sub-item 
        \end{enumerate}
    \end{enumerate}

    % you can customize the symbols before text
    \begin{itemize}
        \item first item 
        \item[$\rightarrow$] second item -------------->        output is ->second item     
        \item[] 3rd item ----------------->     empty [] gives blank space. output is  3rd item   
        \begin{itemize}
            \item First sub-item 
            \item second sub-item 
        \end{itemize}
    \end{itemize}

% you can further customize the lists with "enumitem" package
\usepackage{enumitem}
% use labels such as "\arabic*)" ,"\Roman*", "(\alpha*)","\roman*."
\begin{enumerate}{label=\arabic*)}
    \item first item 
    \item second item 
\end{enumerate}



% ==============other minor commands
% horizontal space , specify the length of space in {}
\hspace{length}

% vertical spaces
\vspace{length}

% boxes
\fbox{text}

% creating horizontal lines
\hrulefill %fills the rest of the line with horizontal line
\dotfill %fills the rest of the line with dotted line
\rule{\linewidth}{0.4pt} % fisrt argument is width and second one is thickness


% importing images
\usepackage{graphicx}
\includegraphics[width=8cm,height=4cm]{mylogo}


% create various headers with different font sizes
\section{first section}
\subsection{first subsection}
\subsubsection{first subsection}


% hyperlinks
\usepackage{hyperref}
\ref{}
\hyperref[label]{link text}
\ref*{}  %to make references without hyperlinks
\href{url}{text} % to create internet hyperlinks
